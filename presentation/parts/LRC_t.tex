\subsection{Definitions and examples for the proof}
\begin{frame}{Recovery graph}

        Assume every coordinate $i$ has $t$ disjoint recovering sets $\R_i^1, \dots, \R_i^t$, each of size $r$, where $\R_i^j \subset \left[n \right] \setminus i$. \pause
        \begin{block}{Definition}
            The \textbf{recovery graph} of a $(n,k,r,t)$ LRC code $\mathcal{C}$ is a directed graph $G=(V,E)$ where: \pause
            \begin{itemize}
                \item $V = \left[n \right]$. (Vertices $\leftrightarrow$ coordinates of $\mathcal{C}$). \pause
                \item $(i,j) \in E \iff j \in \R_i^\ell$ for some $\ell \in \left[ t \right]$.\\ \pause
                There is an edge $i \rightarrow j$ if $j$ is in a recovering set of $i$.\\ \pause
            \end{itemize}    
                Note that $N(i) = \bigcup_{\ell = 1}^{t} \R_i^\ell$
        \end{block}
    \end{frame}    
       
    \begin{frame}
    Recovery graph for the $(9,4,2)$-LRC code. \pause \\~\\
    
    Recall: $\A = \{ A_1 = \{1,3,9\}, A_2 = \{2,6,5 \}, A_3 = \{4,12,10 \} \}$ \pause \\~\\
    
        \begin{center}
        \begin{tikzpicture}
        
            \tikzset{vertex/.style = {shape=circle,draw,minimum size=1.5em}}	
            \tikzset{edge/.style = {->,> = latex'}}
            
	        \node[vertex] (n1) at (0,0) {$1$};
	        \node[vertex] (n3) at (2,0) {$3$};
	        \node[vertex] (n9) at (1,-2) {$9$};
	        
	        \node[vertex] (n6) at (3,-2) {$6$};
	        \node[vertex] (n2) at (5,-2) {$2$};
	        \node[vertex] (n5) at (4,0) {$5$};

	        \node[vertex] (n12) at (6,0) {$12$};
	        \node[vertex] (n4)  at (8,0) {$4$};
	        \node[vertex] (n10) at (7,-2) {$10$};
	        
	        \draw[edge] (n1) to [bend left=10] (n3);
	        \draw[edge] (n1) to [bend left=10] (n9);
	        \draw[edge] (n3) to [bend left=10] (n1);
	        \draw[edge] (n3) to [bend left=10] (n9);
	        \draw[edge] (n9) to [bend left=10] (n1);
	        \draw[edge] (n9) to [bend left=10] (n3);
	        
	        \draw[edge] (n2) to [bend left=10] (n5);
	        \draw[edge] (n2) to [bend left=10] (n6);
	        \draw[edge] (n5) to [bend left=10] (n2);
	        \draw[edge] (n5) to [bend left=10] (n6);
	        \draw[edge] (n6) to [bend left=10] (n2);
	        \draw[edge] (n6) to [bend left=10] (n5);
	        
	        \draw[edge] (n4) to [bend left=10] (n10);
	        \draw[edge] (n4) to [bend left=10] (n12);
	        \draw[edge] (n10) to [bend left=10] (n4);
	        \draw[edge] (n10) to [bend left=10] (n12);
	        \draw[edge] (n12) to [bend left=10] (n4);
	        \draw[edge] (n12) to [bend left=10] (n10);
	
	        %\draw[red, dashed] (1, 2) -- (1, -2);
	            
        \end{tikzpicture}
        \end{center}
    \end{frame}
    
     \begin{frame}
        Color the edges with $t$ distinct colors to differenciate recovering sets. \pause \\~\\
        
        Let $F$ be a coloring function of the edges:
        $$
            \begin{array}{ccccc}
                F: & E(G) & \longrightarrow & [t]  & \\
                   &(i,j) & \longmapsto     & \ell & \mbox{iff } j \in \R_i^\ell
            \end{array}
        $$ \pause \\~\\
        
    Remark: the out-degree of any vertex $i \in V$ is $\sum_\ell \vert \R_i^\ell \vert = tr$, and the edges leaving $i$ are colored in $t$ colors.
    
    \end{frame}    
    
    \begin{frame}
    Recovery graph for the $(12,4,\{2,3\})$-LRC code with edge coloring.
    
    Recall: 
    
    $\mathcal{A} = \left\lbrace  \left\lbrace 1, 5, 12 , 8 \right\rbrace, \left\lbrace 2 , 10 , 11 , 3 \right\rbrace , \left\lbrace 4 , 7 , 9 , 6 \right\rbrace \right\rbrace$
        
     $\mathcal{A'} = \left\lbrace  \left\lbrace 1 , 3 , 9 \right\rbrace, \left\lbrace 2 , 6 , 5 \right\rbrace , \left\lbrace 4 , 12 , 10 \right\rbrace , \left\lbrace 7 , 8 , 11 \right\rbrace \right\rbrace$
     \begin{center}
        \begin{tikzpicture}
            \tikzset{vertex/.style = {shape=circle,draw,minimum size=1.5em}}	
            \tikzset{edge/.style = {->,> = latex',red}}
            \tikzset{edge2/.style = {->,> = latex',blue}}
            
            \def \radius {2.5cm}
            \def \n {12}
            
	        \node[vertex] (n1) at ({360/\n * (0)}:\radius) {$1$};
	        \node[vertex] (n3) at ({360/\n * (1)}:\radius) {$3$};
	        \node[vertex] (n9) at ({360/\n * (2)}:\radius) {$9$};
	        
	        \node[vertex] (n2) at ({360/\n * (3)}:\radius) {$2$};
	        \node[vertex] (n6) at ({360/\n * (4)}:\radius) {$6$};
	        \node[vertex] (n5) at ({360/\n * (5)}:\radius) {$5$};
	        
	        \node[vertex] (n4) at ({360/\n * (6)}:\radius) {$4$};
	        \node[vertex] (n12)  at ({360/\n * (7)}:\radius) {$12$};
	        \node[vertex] (n10) at ({360/\n * (8)}:\radius) {$10$};
	        
	        \node[vertex] (n7) at ({360/\n * (9)}:\radius) {$7$};
	        \node[vertex] (n8)  at ({360/\n * (10)}:\radius) {$8$};
	        \node[vertex] (n11) at ({360/\n * (11)}:\radius) {$11$};
	        
	        \draw[edge] (n1) to [bend left=10] (n3);
	        \draw[edge] (n1) to [bend left=10] (n9);
	        \draw[edge] (n3) to [bend left=10] (n1);
	        \draw[edge] (n3) to [bend left=10] (n9);
	        \draw[edge] (n9) to [bend right=30] (n1);
	        \draw[edge] (n9) to [bend left=10] (n3);
	        
	        \draw[edge] (n2) to [bend left=10] (n5);
	        \draw[edge] (n2) to [bend left=10] (n6);
	        \draw[edge] (n5) to [bend right=30] (n2);
	        \draw[edge] (n5) to [bend left=10] (n6);
	        \draw[edge] (n6) to [bend left=10] (n2);
	        \draw[edge] (n6) to [bend left=10] (n5);
	        
	        \draw[edge] (n4) to [bend left=10] (n10);
	        \draw[edge] (n4) to [bend left=10] (n12);
	        \draw[edge] (n10) to [bend right=30] (n4);
	        \draw[edge] (n10) to [bend left=10] (n12);
	        \draw[edge] (n12) to [bend left=10] (n4);
	        \draw[edge] (n12) to [bend left=10] (n10);
	        
	        \draw[edge] (n7) to [bend left=10] (n8);
	        \draw[edge] (n7) to [bend left=10] (n11);
	        \draw[edge] (n8) to [bend left=10] (n7);
	        \draw[edge] (n8) to [bend left=10] (n11);
	        \draw[edge] (n11) to [bend left=10] (n8);
	        \draw[edge] (n11) to [bend right=30] (n7);
	        
	        \draw[edge2] (n1) to [bend left=10] (n5);
	        \draw[edge2] (n1) to [bend left=10] (n12);
	        \draw[edge2] (n1) to [bend right=10] (n8);
	        
	        \draw[edge2] (n5) to [bend left=10] (n1);
	        \draw[edge2] (n5) to [bend left=10] (n12);
	        \draw[edge2] (n5) to [bend left=10] (n8);
	        
	        \draw[edge2] (n12) to [bend right=20] (n5);
	        \draw[edge2] (n12) to [bend left=10] (n1);
	        \draw[edge2] (n12) to [bend left=10] (n8);
	        
	        \draw[edge2] (n8) to [bend left=10] (n5);
	        \draw[edge2] (n8) to [bend left=10] (n12);
	        \draw[edge2] (n8) to [bend left=20] (n1);
	        
	        
	        \draw[edge2] (n2) to [bend right=20] (n3);
	        \draw[edge2] (n2) to [bend left=10] (n10);
	        \draw[edge2] (n2) to [bend left=10] (n11);
	        
	        \draw[edge2] (n3) to [bend left=10] (n2);
	        \draw[edge2] (n3) to [bend left=10] (n10);
	        \draw[edge2] (n3) to [bend right=20] (n11);
	        
	        \draw[edge2] (n10) to [bend left=10] (n3);
	        \draw[edge2] (n10) to [bend left=10] (n2);
	        \draw[edge2] (n10) to [bend left=10] (n11);
	        
	        \draw[edge2] (n11) to [bend left=10] (n3);
	        \draw[edge2] (n11) to [bend left=10] (n10);
	        \draw[edge2] (n11) to [bend left=10] (n2);

	        
	        \draw[edge2] (n4) to [bend right=20] (n6);
	        \draw[edge2] (n4) to [bend left=10] (n7);
	        \draw[edge2] (n4) to [bend left=10] (n9);
	        
	        \draw[edge2] (n6) to [bend left=10] (n4);
	        \draw[edge2] (n6) to [bend left=10] (n7);
	        \draw[edge2] (n6) to [bend right=20] (n9);
	        
	        \draw[edge2] (n7) to [bend left=10] (n6);
	        \draw[edge2] (n7) to [bend left=10] (n4);
	        \draw[edge2] (n7) to [bend left=10] (n9);
	        
	        \draw[edge2] (n9) to [bend left=10] (n6);
	        \draw[edge2] (n9) to [bend left=10] (n7);
	        \draw[edge2] (n9) to [bend left=10] (n4);
	            
        \end{tikzpicture}
        \end{center}
    \end{frame}
    
    
    \begin{frame}
    Recovery graph for the $(12,4,\{2,3\})$-LRC code with edge coloring.
    
    Recall: 

    $
        \mathcal{A} = \left\lbrace
        \only<5>{\textcolor{blue}}{\left\lbrace 1, 5, 12 , 8 \right\rbrace},
        \only<6>{\textcolor{blue}}{\left\lbrace 2 , 10 , 11 , 3 \right\rbrace},
        \only<7>{\textcolor{blue}}{\left\lbrace 4 , 7 , 9 , 6 \right\rbrace}
        \right\rbrace
    $
    
     $ 
         \mathcal{A'} = \left\lbrace
         \only<1>{\textcolor{red}}{\left\lbrace 1 , 3 , 9 \right\rbrace},
         \only<2>{\textcolor{red}}{\left\lbrace 2 , 6 , 5 \right\rbrace},
         \only<3>{\textcolor{red}}{\left\lbrace 4 , 12 , 10 \right\rbrace},
         \only<4>{\textcolor{red}}{\left\lbrace 7 , 8 , 11 \right\rbrace}
         \right\rbrace
     $   
        \begin{center}
        \begin{tikzpicture}
            \tikzset{vertex/.style = {shape=circle,draw,minimum size=1.5em}}	
            \tikzset{edge1/.style = {-,> = latex',red}}
            \tikzset{edge2/.style = {-,> = latex',blue}}
            
            \def \radius {2.5cm}
            \def \n {12}
            
	        \node[vertex] (n1) at ({360/\n * (0)}:\radius) {$1$};
	        \node[vertex] (n3) at ({360/\n * (1)}:\radius) {$3$};
	        \node[vertex] (n9) at ({360/\n * (2)}:\radius) {$9$};
	        
	        \node[vertex] (n2) at ({360/\n * (3)}:\radius) {$2$};
	        \node[vertex] (n6) at ({360/\n * (4)}:\radius) {$6$};
	        \node[vertex] (n5) at ({360/\n * (5)}:\radius) {$5$};
	        
	        \node[vertex] (n4) at ({360/\n * (6)}:\radius) {$4$};
	        \node[vertex] (n12)  at ({360/\n * (7)}:\radius) {$12$};
	        \node[vertex] (n10) at ({360/\n * (8)}:\radius) {$10$};
	        
	        \node[vertex] (n7) at ({360/\n * (9)}:\radius) {$7$};
	        \node[vertex] (n8)  at ({360/\n * (10)}:\radius) {$8$};
	        \node[vertex] (n11) at ({360/\n * (11)}:\radius) {$11$};
	        
	        \only<1,8>{
	        \draw[edge1] (n1) to (n3);
	        \draw[edge1] (n1) to (n9);
	        \draw[edge1] (n3) to (n9);
	        }
	        
	        \only<2,8>{
	        \draw[edge1] (n2) to (n6);
	        \draw[edge1] (n2) to (n5);
	        \draw[edge1] (n5) to (n6);
	        }
	        
	        \only<3,8>{
	        \draw[edge1] (n4) to (n10);
	        \draw[edge1] (n4) to (n12);
	        \draw[edge1] (n12) to (n10);
	        }
	        
	        \only<4,8>{
	        \draw[edge1] (n7) to (n8);
	        \draw[edge1] (n7) to (n11);
	        \draw[edge1] (n8) to (n11);
	        }
	        
	        \only<5,8>{
	        \draw[edge2] (n1) to (n5);
	        \draw[edge2] (n1) to (n12);
	        \draw[edge2] (n1) to (n8);
	        \draw[edge2] (n5) to (n12);
	        \draw[edge2] (n5) to (n8);
	        \draw[edge2] (n8) to (n12);
	        }
	        
	        \only<6,8>{
	        \draw[edge2] (n2) to (n10);
	        \draw[edge2] (n2) to (n11);
	        \draw[edge2] (n2) to (n3);
	        \draw[edge2] (n10) to (n11);
	        \draw[edge2] (n10) to (n3);
	        \draw[edge2] (n11) to (n3);
	        }
	        
	        \only<7,8>{
	        \draw[edge2] (n4) to (n7);
	        \draw[edge2] (n4) to (n9);
	        \draw[edge2] (n4) to (n6);
	        \draw[edge2] (n7) to (n9);
	        \draw[edge2] (n7) to (n6);
	        \draw[edge2] (n9) to (n6);
	        }
	        
    \end{tikzpicture}
    \end{center}
\end{frame}

\subsection{Proof for the rate bound}
\begin{frame}{Proof for the rate bound}
To prove bound on max. rate:
\begin{enumerate}
    \item Construct set $U$ of coordinates that can be recovered from $\overline{U} := [n]\setminus U$.
    \item Compute lower bound on $\card{U}$ $ \quad \longrightarrow \quad $ upper bound on $\card{\overline{U}}$
    \item u. bound on $\card{\overline{U}}$ $\rightarrow$ u. bound on $k$ $\rightarrow$ u. bound on max. rate 
\end{enumerate}
\end{frame}