\chapter{Context and motivation}

In recent years the explosion in the volumes of data being stored online has resulted in distributed storage systems transitioning to erasure coding based schemes in order to ensure reliability with low storage overheads. On such a massive scale, unreachable or failed servers are no longer an exception but a regular occurrence and recovery from such events has to be done efficiently.

Classical erasure correcting codes guarantee that data can be recovered if a bounded number of codeword coordinates is erased. However recovering data typically involves accessing all surviving coordinates. 

In recent years Locally Recoverable Codes (LRC) emerged as the codes of choice for many such scenarios and have been implemented in a number of large scale systems (see \cite{azure} and \cite{hadoop}). LRC codes have the property that a symbol of the codeword can be recovering accessing few other symbols of the codeword (called the \textit{recovering set}).

Symbols can have more than one recovering set, and having more than one recovering set is beneficial in practice because it enables more users to access a given portion of data, thus enhancing data availability in the system.

Data storage applications require codes with small redundancy, low locality for information coordinates, large distance, and low locality for parity coordinates.