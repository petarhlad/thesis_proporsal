\chapter{Coding Theory Preliminaries}

\section{Linear codes}
Let $\FF_q$ be the finite field with $q$ elements, and consider $\FF_q^n$ the vector space of dimension $n$ over $\FF_q$.

\begin{defn}[Linear Code]
An $(n,k)$ linear code over $\FF_q$ is a subspace $\C \subset \FF_q^n$ with $\dim (\C) = k$. $\C$ is said to have length $n$ and dimension $k$. Every element $c \in \C$ is called a \textit{codeword}.
\end{defn}

Consider an $(n,k)$ linear code $\C$ over $\FF_q$, and let $\mathbf{x} = x_1 \dots x_n$, $\mathbf{y} = y_1 \dots y_n$ two codewords of $\C$.

\begin{defn}[Hamming Distance]
The Hamming distance between two vectors $x = x_1 \dots x_n$ and $y = y_1 \dots y_n$ is the number of coordinates where they differ, and is denoted by dist$(x,y)$.
\end{defn}

\begin{defn}[Hamming Weight]
The Hamming Weight of a vector $x = x_1 \dots x_n$ is the number of nonzero coordinates of $x$ and is denoted by wt$(x)$.
\end{defn}

\begin{remk}
For a linear code $\C$ and any two codewords $x,y \in \C$:
\begin{equation}
\dist(x,y) = \wt(x-y)
\end{equation}
\end{remk}

\begin{defn}[Minimum distance]
A code $\C$ has minimum distance $d$ if any two codewords differ in at least $d$ coordinates.
\begin{equation}
d = \min \dist (x,y) = \min \wt (x-y), \quad x,y \in \C, x \neq y
\end{equation}
\end{defn}

A linear code of length $n$, dimension $k$, and minimum distance $d$ will be called an $[n,k,d]$ linear code.

There are several known and well studied bounds on the size of a code considering its length and distance. One that will be very important in this work is the Singleton Bound.

\begin{thm}[Singleton Bound] If $\C$ is an $[n,k,d]$ code, then 
\begin{equation}
d \leq n - k + 1
\end{equation}
\end{thm}

\begin{defn}[MDS Codes]
A Maximum Distance Separable code $\C$ is a code that attains the Singleton bound with equality. That is, a code s.t.
\begin{equation}
d = n - k + 1
\end{equation}
\end{defn}