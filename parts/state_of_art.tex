\chapter{State of the Art}

\section{Definition of LRC codes}

Consider a linear $[n,k,d]_q$ code $\C \subset \FF_q^n$, where $q$ is a prime power. We say that the $i$-th coordinate of $\C$ has locality $r$, if the value at this coordinate can be recovered from accessing some other $r$ coordinates of $\C$. We say that the code $\C$ has locality $r$ if every symbol of the codeword $x \in \C$ can be recovered from a subset of $r$ other symbols of $x$.

\begin{defn}[LRC Codes]
A code $\C \subset \FF_q^n$ is a \textit{locally recoverable code} (LRC) with locality $r$ if for every $i \in [n]$ there exists a subset $\R_i \subset [n] \setminus \{i\}$, $\card{\R_i} \leq r$ and a map $\phi_i$ such that for every codeword $\x \in \C$ we have
\begin{equation}
\x_i = \phi_i(\{\x_j, \ j \in \R_i \})
\end{equation}
This definition can be also rephrased as follows. Given $a \in \FF_q$ consider the sets of codewords
\[\C(i,a) = \{ x \in \C : x_i = a\}, \quad i \in [n]\]
The code $\C$ is said to have locality $r$ if for every $i \in [n]$ there exists a subset ${\R_i \subset [n] \setminus \{i\}}, \ \card{\R_i} \leq r$ such that the restrictions of the sets $\C(i,a)$ to the coordinates in $\R_i$ for different $a$ are disjoint:
    \begin{equation}
        \C_{I_i}(i,a) \cap \C_{I_i}(i,a') = \emptyset, \quad a \neq a'
    \end{equation}
The subset $I_i$ is called a \textit{recovering set} for the symbol $x_i$.
\end{defn}

\begin{defn}[t-LRC Codes]
A code $\C$ is said to have $t$ disjoint recovering sets if for every $i \in [n]$ there are pairwise disjoint subsets $R_i^1, ..., R_i^t \subset [n] \setminus \{i\}$ such that for all $j =1, ..., t$ and every pair of symbols $a,a' \in \FF_q, \ a \neq a'$
\begin{equation}
\C(i,a)_{R_i^j} \cap \C(i,a')_{R_i^j} = \emptyset
\end{equation}

\end{defn}

For linear LRC codes, the relation between a symbol $i$ and its recovering set $I_i$ is linear. Thus, any symbol in $I_i \cup \{i\}$ can be recovered from the remaining symbols. We then call $I_i \cup \{i\}$ a \textit{repair group}.

\section{Bounds on parameters of LRC codes}
\citeauthor*{GHSY12} proved in \cite{GHSY12} the following bounds:
\begin{thm}
Let $\C$ be an $(n,k,r)$ LRC code. The rate of $\C$ satisfies
\begin{equation}
    \frac{k}{n} \leq \frac{r}{r+1}
\end{equation}

\noindent The minimum distance of $\C$ satisfies:
\begin{equation}\label{eq:opt_lrc}
d \leq n -k - \left\lceil \frac{k}{r} \right\rceil + 2
\end{equation}


\end{thm}


\begin{thm}[\cite{RPDV16, wang14}]

For $(n,k,r,t)$ LRC codes with $t \geq 2$ disjoint recovering sets:
\begin{equation}\label{eq:opt_lrc-t}
    d \leq n-k + 2 - \left\lceil \frac{t(k-1)+1}{t(r-1)+1} \right\rceil
\end{equation}
\end{thm}

We will refer to codes attaining the bound \ref{eq:opt_lrc} (the bound \ref{eq:opt_lrc-t} in case $t \geq 2$) as optimal LRC codes.

In \cite{bounds_on_LRC}, \citeauthor{bounds_on_LRC} found new bounds on the distance and rate of LRC codes as well as assymptotic bounds.

\begin{thm}\label{thm:lrc_rate}
Let $\C$ be an $(n,k,r,t)$ LRC code with $t$ disjoint recovering sets of size $r$. Then the rate of $C$ satisfies
\begin{equation}\label{eq:rate_lrc}
\frac{k}{n} \leq \frac{1}{\prod_{j=1}^t(1+\frac{1}{jr})}
\end{equation}
The minimum distance of $C$ is bounded above as follows:
\begin{equation}\label{eq:ass_rate_lrc}
d \leq n - \sum_{i=0}^t \left\lfloor \frac{k-1}{r^i} \right\rfloor
\end{equation}
\end{thm}

\begin{equation}
R_q (r,\delta) \geq 1 - \min_{0 < s \leq 1} \left\lbrace \frac{1}{r+1}\log_q ((1+(q-1)s)^{r+1}+(q-1)(1-s)^{r+1})-\delta \log_q s \right\rbrace
\end{equation}

\section{Algebraic Geometric Codes}

A family of optimal LRC codes was described by \citeauthor{optimal_LRC} in \cite{optimal_LRC} which are subcodes of Reed-Solomon codes, considering a polynomial that is constant on each part of a partition of the evaluation points set. The length of these codes is upper bounded by $q$ (same as RS codes). Then an algebraic geometric approach of an optimal LRC code is obtained from maps of degree $r+1$ from $\P^1_{\FF_q}$ to $\P^1_{\FF_q}$.

In \cite{LRC_on_alg_curves}, the authors generalized this idea and constructed LRC codes from morphisms on algebraic curves (Hermitian curves and Garcia-Stichtenoth curves). In \cite{LRC_on_alg_curves2} the authors expand on the constructions of \cite{LRC_on_alg_curves} to produce families of LRC codes coming from a larger variety of curves, as well as from higher-dimensional varieties. They construct optimal LRC codes with code length larger than the size of the alphabet (e.g. $n = q^2 + 2$ and $n = q^2 - 1$).

%Let $X$ be a nonsingular irreducible projective curve over $K=\FF_q$ with genus $g$ and let $K(X)$ be the function field of $X$. For a divisor $G$ on $X$ define the vector space $L(G) := \{ f \in K(X) \vert div(f) + G > 0\} \cup \{0\}$.
%
%Assume $P_1, ..., P_n$ are rational points on the curve $X$ and let $D = P_1 + ... + P_n$. Assume $G$ is a divisor on $X$ with rational points and support disjoint from $D$. Also assume that $2g - 2 < deg(G) < n$.
%
%\begin{defn}
%The linear code $\C(D,G)$ over $\FF_q$ is the image of the linear map $\alpha: L(G) \rightarrow \FF_q^n$ where $\alpha(f) = (f(P_1), ... , f(P_n))$.
%\end{defn}
%
%\begin{thm}
%The code $\C(D,G)$ has parameters $[n,k,d]_q$ with
%$$n = deg(D), \quad k = deg(G)-g+1, \quad d \geq d^\ast = n - deg(G)$$
%\end{thm}

\section{Cyclic and Binary LRC codes}
Binary error correcting codes are of special interest due to practical reasons.

In \cite{binary_LRC_decoding2} authors show how the presence of locality within a binary cyclic code can be exploited to improve decoding performance and reduce decoding complexity. They approach the problem with ordered statistics decoding (OSD) method and with trellis decoding.

In \cite{binary_constructions}, authors consider linear cyclic codes with the locality property. They focus on optimal cyclic codes that arise from the construction of RS like LRC codes in \cite{optimal_LRC}, and give a characterization of these codes in terms of their zeros, and observe that there are many equivalent ways of constructing optimal cyclic LRC codes oven a given field.

\section{List decoding}
A code of length $n$ is called $(\tau,\ell)$-list decodable if the Hamming sphere of radius $\tau$ centered at any vector $v$ of length $n$ always contains at most $\ell$ codewords $c \in \C$. It was shown by \citeauthor{list_decoding} in \cite{list_decoding} that any code of length $n$ and distance $d$ is $(\tau_J, \ell)$-list decodable where $\tau_J = n - \sqrt{n(n-d)}$ is the Johnson radius and $\ell \in poly(n)$.

It was recently shown by \citeauthor{list_decoding_LRC} in \cite{list_decoding_LRC} that the list decoding radius of certain LRC codes exceed the Johnson radius and give a general list decoding algorithm. The complexity of the algorithm is polynomial in $n$ when the number of repairing groups is constant, otherwise it grows exponentially.