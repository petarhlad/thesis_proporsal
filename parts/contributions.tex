\chapter{Open Problems}
Alexander Barg is one of the main contributors in the topic of LRC codes. When discussing with him about open problems in the field, he said that the main problems have already been worked on, and the remaining problems are non-trivial and hard. Therefore, specific problems will be studied in this PhD thesis.

\noindent Problems that will be studied:
\begin{itemize}[align = left, leftmargin=*]
\item[\textbf{Improvement of binary LRC codes decoding:}] Following the work in \cite{binary_LRC_decoding2}, decoding of binary LRC codes that are not cyclic will be studied.
\item[\textbf{New constructions of LRC codes on algebraic curves:}] Following the work in \cite{LRC_on_alg_curves} and \cite{LRC_on_alg_curves2}, new constructions of LRC codes over algebraic varieties will be searched, looking for optimal LRC codes with small field size.
\item[\textbf{Improvement of bounds for LRC-$t$ codes:}] The lower bound \ref{eq:rate_lrc} appears to be far from tight. \citeauthor{bounds_on_LRC_t} in \cite{bounds_on_LRC_t} said they believe that the rate $\left(\frac{r}{r+1}\right)^t$ is the largest possible for a LRC-$t$ code as long as $t$ is not too large (e.g. $t \in O(\log n)$). This rate can be achieved constructing a $t$-fold power of the binary $(r+1,r)$ single-parity-check code.

Theorem \ref{thm:lrc_rate} is proved applying probabilistic method techniques on the properties of a graph. The problem of optimizing the bound of the rate of LRC-t codes will be studied, and one of the ways could be following a similar proof considering some restrictions that were not considered in \cite{bounds_on_LRC}.

\item[\textbf{List decoding of LRC codes:}] the problem will be studied to search for new families of LRC codes that could be list decoded beyond the Johnson radius.

\end{itemize}