\chapter{Work Plan}
\section{Goals}
The goal of this PhD program is to make contributions in the field of Coding Theory, specifically in Locally Recoverable Codes. To do it, a number of open problems are detailed in section \ref{sec:open_probl}, that will be studied.

Alexander Barg is one of the main contributors in the topic of LRC codes. When discussing with him about open problems in the field, he said that the main problems have already been worked on, and the remaining problems are non-trivial and hard. Therefore, specific problems will be studied in this PhD thesis.

\section{Open Problems}
\label{sec:open_probl}
\begin{itemize}[align = left, leftmargin=*]
\item[\textbf{Improvement of binary LRC codes decoding:}] Following the work in \cite{binary_LRC_decoding2}, decoding of binary LRC codes that are not cyclic will be studied.
\item[\textbf{New constructions of LRC codes on algebraic curves:}] Following the work in \cite{LRC_on_alg_curves} and \cite{LRC_on_alg_curves2}, new constructions of LRC codes over algebraic varieties will be searched, looking for optimal LRC codes with small field size.
\item[\textbf{Improvement of bounds for LRC-$t$ codes:}] The lower bound \ref{eq:rate_lrc} appears to be far from tight. \citeauthor{bounds_on_LRC_t} in \cite{bounds_on_LRC_t} said they believe that the rate $\left(\frac{r}{r+1}\right)^t$ is the largest possible for a LRC-$t$ code as long as $t$ is not too large (e.g. $t \in O(\log n)$). This rate can be achieved constructing a $t$-fold power of the binary $(r+1,r)$ single-parity-check code.

Theorem \ref{thm:lrc_rate} is proved applying probabilistic method techniques on the properties of a graph. The problem of optimizing the bound of the rate of LRC-t codes will be studied, and one of the ways could be following a similar proof considering some restrictions that were not considered in \cite{bounds_on_LRC}.

\item[\textbf{List decoding of LRC codes:}] the problem will be studied to search for new families of LRC codes that could be list decoded beyond the Johnson radius.
\end{itemize}


\section{Time Plan and Methodology}
The work and research on the open problems is planned for a temporal span of 3 years.

\begin{itemize}[align = left, leftmargin=*]
\item[\textbf{First year:}] Initial research and open problems statement
\item[\textbf{Second year:}] Research on stated problems
\item[\textbf{Third year:}] Writing
\end{itemize}

\subsection*{First Year}
During the first year several activities have been done to determine the state of the art of Locally Recoverable Codes, and state the problems that will be the object of study in this PhD thesis.

\paragraph*{Master's Degree}
Obtention of Master's Degree in Advanced Mathematics and Mathematical Engineering, attending specific courses related to the problems that will be worked on. Courses: Coding Theory, Commutative Algebra, Algebraic Geometry, Combinatorics and Graph Theory.

\paragraph*{Stays}
Two week stay in University of Maryland as a visitor student invited by Prof. Alexander Barg. In Those two weeks collaborating with Prof. Barg, a better understanding on the state of the art and which are the interesting problems to solve.

\paragraph*{Seminars}
Algebraic Geometry Seminar, two sessions per week. Following:
\begin{itemize}
\item Book on algebraic curves \cite{Fulton}
\item Lecture notes on algebraic geometry and algebraic curves \cite{gathmann, gathmann_curves}
\end{itemize}

\paragraph*{Self-Learning} Followed:
\begin{itemize}
\item Book on Probabilistic Method \cite{prob_method}
\end{itemize}

\subsection*{Second Year}
During the second year, the main focus will be on the stated problems.

\paragraph*{Seminars}
\begin{itemize}
\item Algebraic Geometric Codes, two sessions per week. Following: Books on Algebraic Geometric Codes \cite{stichtenoth, vladut}
\item Probabilistic method, two sessions per week. Following: Book on Probabilistic Method \cite{prob_method}
\end{itemize}

\paragraph*{Self-Learning} Will follow:
\begin{itemize}
\item Book on Coding Theory \cite{mcwilliams}
\end{itemize}

\subsection*{Third Year}
During the third year, the goals will be to conclude with the research on the problems and to write the PhD thesis.
