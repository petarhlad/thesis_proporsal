\chapter{Open Problems}
\cite{MR_LRC}
\section{Constructions of binary LRC codes}
\cite{binary_constructions}

\section{Constructions of codes on algebraic curves}
\cite{LRC_on_alg_curves, LRC_on_alg_curves2, LRC_fibers}

\section{Bounds on codes with locality}
\cite{combinatorial_bounds}
\cite{bounds_on_LRC}
\cite{bounds_on_LRC_t}

\begin{thm}[\cite{bounds_on_LRC}]
Let $\C$ be an $(n,k,r,t)$ LRC code with $t$ disjoint recovering sets of size $r$. Then the rate of $C$ satisfies
\begin{equation}
\frac{k}{n} \leq \frac{1}{\prod_{j=1}^t(1+\frac{1}{jr})}
\end{equation}
The minimum distance of $C$ is bounded above as follows:
\begin{equation}
d \leq n - \sum_{i=0}^t \left\lfloor \frac{k-1}{r^i} \right\rfloor
\end{equation}
\end{thm}

\section{List decoding of LRC codes}


A code of length $n$ is called $(\tau,\ell)$-list decodable if the Hamming sphere of radius $\tau$ centered at any vector $v$ of length $n$ always contains at most $\ell$ codewords $c \in \C$. It was shown by \citeauthor{list_decoding} in \cite{list_decoding} that any code of length $n$ and distance $d$ is $(\tau_J, \ell)$-list decodable where $\tau_J = n - \sqrt{n(n-d)}$ and $\ell \in poly(n)$.

It was recently shown by \citeauthor{list_decoding_LRC} in \cite{list_decoding_LRC} that 